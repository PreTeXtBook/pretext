% Sage CLI Quick Reference
% (c) 2022 by Thomas W.\ Judson,
% Licensed with the GNU Free Documentation License (GFDL)
%   http://www.gnu.org/copyleft/fdl.html
%
%  History
%
%    2012-06-15  Initial version based on Sage 9.4
%
%
\documentclass{article}
\usepackage{graphicx}  
\usepackage[landscape]{geometry}
% presumes xelatex, so other character sets behave better
\usepackage[xetex]{color}
\usepackage{url}
\usepackage{multicol}
\usepackage{amsmath}
\usepackage{amsfonts}
\usepackage{textcomp}
\newcommand{\ex}{\color{blue}}
\newcommand{\apost}{\textquotesingle}
\newcommand{\warn}{\bf\color{red}}
\pagestyle{empty}
\advance\topmargin-.9in
\advance\textheight2in
\advance\textwidth3.0in
\advance\oddsidemargin-1.45in
\advance\evensidemargin-1.45in
\parindent0pt
\parskip2pt
% Section break, dictates column widths?
\newcommand{\hr}{\centerline{\rule{3.5in}{1pt}}}
% Adjust gap to affect spacing, page count
\newcommand{\sect}[1]{\hr\par\vspace*{2pt}\textbf{#1}\par}
% Mandatory indentation on subsidiary lines
\newcommand{\skipin}{\hspace*{12pt}}
% notation shortcut
\newcommand{\Z}{\mathbb{Z}}

\begin{document}
\begin{multicols*}{3}
\begin{center}
\textbf{PreTeXt Authoring Quick Reference}\\
% change review date iff there is an actual
% review, not just some isolated update
Version 1.0, reviewed 2022-07-27\\
T.\ W.\ Judson and others??? \\
GNU Free Document License, extend for your own use. \\
For more details, see \url{https://pretextbook.org/doc/guide/html/}
\end{center}
% backup over center environment gap
\vspace{-2ex}
%*********************************************
\sect{PreTeXt Documents}
For an article
\footnotesize{
\begin{verbatim}
<?xml version="1.0" encoding="UTF-8"?>
<pretext>
  <article>
    <title>Hello World!</title>
    <p>This is a PreTeXt document.</p>
  </article>
</pretext>
\end{verbatim}
or a book
\begin{verbatim}
<?xml version="1.0" encoding="UTF-8"?>
<pretext>
  <book>
    <title>Hello World!</title>
    
    <chapter>
    	<title>My Great Chapter</title>
    	<p>This is a PreTeXt document.</p>
   </chapter>
   
  </book>
</pretext>
\end{verbatim}
}
%

\sect{Structure of a PreTeXt Document}

PreTeXt documents are structured and may contain divisions such as \texttt{<chapter>} (for books), \texttt{<section>}, \texttt{<subsection>}, and \texttt{<p>} (paragraphs).

{\footnotesize
\begin{verbatim}
<section>
  <title>Mandatory</title>
  <p>First paragraph. </p>

  <p>Second paragraph.</p>
</section>
\end{verbatim}
}

Divisions may contain other divisions. Divisions require a \texttt{<title>}.

{\footnotesize
\begin{verbatim}
<section>
  <title>Mandatory</title>
  <introduction>
    <p>Introductory text. (Optional.)</p>
  </introduction>

  <subsection>
    <title>Mandatory</title>
    <p>Subsection content.</p>
  </subsection>

  <conclusion>
    <p>Concluding text. (Optional.)</p>
  </conclusion>
</section>
\end{verbatim}
}
%*********************************************
%*********************************************
\sect{Blocks}

Besides paragraphs (\texttt{<p>}) the most common object to include in a division, 
 \texttt{<remark>},  \texttt{<example>}, \texttt{<figure>} and \texttt{<table>}.
 
 \sect{Cross-References}

Any element that you place a \texttt{@xml:id} on can become the target of a cross-reference. For example, suppose your source had \texttt{<subsection xml:id="subsection-flowers">} and someplace else you wrote \texttt{<xref ref="subsection-flowers" />}.

\sect{Mathematics in PreTeXt}


Since PreTeXt has robust support for mathematical formulas. Inside the tags that delimit math environments, your code is basically \LaTeX\, with the caveat that you must be careful with \texttt{<}, \texttt{>}, and \texttt{\&} since they are special symbols for XML.  When typing math in your PreTeXt code, use \texttt{\textbackslash{lt}} for  \texttt{<}, \texttt{\textbackslash{gt}} for  \texttt{>}, and \texttt{\textbackslash{amp}} for  \texttt{\&}.

For inline math, wrap things in the \texttt{<m>} tag: $a^2 + b^2 = c^2$ is 
produced by \texttt{<m>a\^{}2 + b\^{}2 = c\^{}2</m>}.



We get displayed equations via the \texttt{<me>} and \texttt{<men>}. (to produce a numbered equation) tags.  The code
{\footnotesize
\begin{verbatim}
<me>
   \frac{d}{dx} \int_1^x \frac{1}{t}\, dt 
</me>
<men xml:id="eqn-ftc">
  \int_a^b f(x)\, dx = F(b) - F(a)
</men>
\end{verbatim}
}
produces
\[
 \frac{d}{dx} \int_1^x \frac{1}{t}\, dt 
 \]
 \begin{equation}
 \int_a^b f(x)\, dx = F(b) - F(a)
 \end{equation}

For a collection of equations all aligned at a designated point, use \texttt{<md>} and \texttt{<mrow>} (\texttt{<mdn>} for numbered equations.). The code
\begin{verbatim}
  <md>
    <mrow>x \amp = r\cos\theta</mrow>
    <mrow>y \amp = r\sin\theta</mrow>
  </md>
  \end{verbatim}
  produces
  \begin{align*}
   x & = r\cos\theta \\
   y & = r\sin\theta.
  \end{align*}


\sect{Images, Figures, sidebyside}

Images can be included using the \texttt{<image>} tag with the \texttt{@source}.  The \texttt{@width} attribute can be used to control the size of the image.  Images can be wrapped inside a \texttt{<figure>}. A \texttt{<figure>} must have a \texttt{<caption>}, and the figure will be numbered.
The \texttt{<sidebyside>} tag provides flexible options for placing several images together or combining figures with subcaptions.
 PreTeXt provides support for authoring with graphics languages such as Asymptote, TikZ, PGF, PSTricks, and xy-pic in addition to using Sage code to describe a plot or image. In most cases output can be obtained as smoothly-scalable SVG images, in addition to other formats like PDF or PNG. For accessibility, every \texttt{<image>} should either have a \texttt{<description>} child.
{ \footnotesize
 \begin{verbatim}
<figure xml:id="figure-spring-mass">
  <image width="60\%" xml:id="spring-mass">
    <description>a mass on a table that is
       attached to a wall with a spring</description>
     <latex-image>
       <xi:include href="tikz/spring-mass.tex"  
         parse="text"/>
      </latex-image>
  </image>
   <caption>A spring-mass system</caption>
</figure>
\end{verbatim}
}
 %*********************************************
\sect{Lists}

The structure of ordered lists (numbered), unordered lists (bullets) and description lists (defined terms) is given by the \texttt{<ol>}, \texttt{<ul>}, \texttt{<dl>} tags (respectively). List items are delimited with the \texttt{<li>} tag.

%*********************************************
\sect{Theorem-Like Elements}

The tags \texttt{<theorem>}, \texttt{<algorithm>}, \texttt{<claim>}, \texttt{<corollary>}, \texttt{<fact>}, \texttt{<identity>}, \texttt{<lemma>}, and \texttt{<proposition>} have the same structure in PreTeXt.
 {\footnotesize
 \begin{verbatim}
<theorem>
    <title>Optional</title>
  <statement>
    <p>Here's the statement of the theorem.</p>
  </statement>

  <proof>
    <p>You don't actually need a proof.</p>
  </proof>
</theorem>
\end{verbatim}
}

\sect{Example-Like Elements}

The tags \texttt{<example>}, \texttt{<problem>}, and \texttt{<question>} have the same structure in PreTeXt.
 {\footnotesize
 \begin{verbatim}
<example>
  <title>Differentiating a polynomial</title>
  <p>The derivative of the function
  <m>f(x) = 3x^5-7x+5</m> is <m>f'(x) = 15x^4-7</m>.</p>
</example>
\end{verbatim}
}


\sect{Axiom-Like Elements}

The tags \texttt{<assumption>}, \texttt{<axiom>}, \texttt{<conjecture>}, \texttt{<heuristic>}, \texttt{<hypothesis>}, and \texttt{<principle>} have the same structure in PreTeXt.
 {\footnotesize
 \begin{verbatim}
<axiom>
    <title>Optional</title>
    <creator>Peano</creator>
  <statement>
    <p>Here's the statement of the axiom.</p>
  </statement>
</axiom>
\end{verbatim}
}

\sect{Remark-Like Elements}

The tags \texttt{<convention>}, \texttt{<insight>}, \texttt{<note>}, \texttt{<observation>}, \texttt{<remark>}, and \texttt{<warning>} have the same structure in PreTeXt.
 {\footnotesize
 \begin{verbatim}
<remark>
  <title>A little remark</title>
  <p>This is a remark.</p>
</remark>
\end{verbatim}
}



 \hr\textbf{Project-Like Elements}


The tags \texttt{<activity>}, \texttt{<exploration>}, \texttt{<investigation>}, and \texttt{<project>} have the same structure in PreTeXt.
 {\footnotesize
 \begin{verbatim}
<project>
  <title>A structured project</title>
  <introduction>
    <p>Here is the introduction.</p>
  </introduction>

  <task>
    <statement>
      <p>The first step to do.</p>
    </statement>
  </task>

  <task>
    <statement>
      <p>The second step to do.</p>
    </statement>
  </task>

  <conclusion>
    <p>A little wrap up.</p>
  </conclusion>
</project>
\end{verbatim}
}


 \sect{Exercises}
 
An \texttt{<exercise>} in the middle of a division, intermixed between theorems and paragraphs and figures. In this case, it is labeled as a ``Checkpoint.''  You can put several \texttt{<exercise>}s as part of an \texttt{<exercises>} element within a division, which is the typical way for creating a collection of exercises together at the end of a division such as a chapter or section.  An \texttt{<exercisegroup>} can group together a collection of exercises that have a set of common instructions.  A specialized division, \texttt{<reading-questions>}, can be used to house \texttt{<exercise>}s designed to test or guide a reader's comprehension of the material in that division.  It is possible to embed WeBWorK exercises into a PreTeXt document

An \texttt{<exercise>} has the following structure.
 {\footnotesize
 \begin{verbatim}
<exercise>
  <statement>
    <p>The <c>statement</c> is mandatory.</p>
  </statement>
  <hint>
    <p>Optional.</p>
  </hint>
  <answer>
    <p>Optional.</p>
  </answer>
  <solution>
    <p>Optional.</p>
  </solution>
</exercise>
\end{verbatim}
}

\sect{Worksheets}

A \texttt{<worksheet>} is a specialized division that can be a child of most divisions and can contain most PreTeXt tags.

\sect{Tables}

Similar to \LaTeX\, PreTeXt provides a \texttt{<table>} tag and a \texttt{<tabular>} tag. The \texttt{<tabular>} tag is used for producing the array of data, while the \texttt{<table>} tag provides the number and title.

%*********************************************

%*********************************************
 \hr\textbf{SageMath Content}

A SageMath cell can be included in a PreTeXt document.

 {\footnotesize
 \begin{verbatim}
<sage>
<input>
2+2
</input>
<output>
4
</output>
</sage>
\end{verbatim}
}

SageMath can be used to created an image in a PreTeXt document.

 {\footnotesize
 \begin{verbatim}
<figure xml:id="fig-sage-cubic">
  <caption>A cubic plotted by SageMath on
  <m>[-3,2]</m></caption>
  <image xml:id="sageplot-cubic" width="50%">
    <description>A cubic function on the interval
    [-3,2]</description>

    <sageplot>
     f(x) = (x-1)*(x+1)*(x-2)
     plot(f, (x, -3, 2), color='blue', thickness=3)
    </sageplot>
  </image>
</figure>
\end{verbatim}
}



\end{multicols*}

\end{document}
